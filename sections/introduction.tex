
\chapter{Introduction}

This experiment is based on the Diplomarbeit of Mathias Heel in 1982 (\textit{Aufbau eines Versuchs im Fortgeschrittenen-Praktikum zur Messung der Lebendauer von Myonen}) and in particular the Staatsexamensarbeit of Joachim Geisb\"{u}sch in 1991 (Verbesserungen zum Experiment) at the University of Mainz's Departement of Physics. This experiment script was re-written in English by Matei Climescu in 2020 from Andreas Winhart's German version from 2001. The full latex source is publicly available at \href{https://github.com/matclim/muonlifetime}{https://github.com/matclim/muonlifetime} for improvements, updates and changes, it is generally advised that students also look at the repository to ensure they have the latest version.



This experiment allows for the determination of the lifetime and magnetic moment of cosmic muons. Students can have a first introduction to elementary particle physics measurements and analysis methods with relatively simple means. 


The muon \Pmuon is an elementary particle often referred to as the electron's `heavy cousin'. It has identical properties to the electron and thus behaves in the same way, if the electron's mass was 207 times larger. The same can be said for the charged tau lepton \Ptauon (sometimes called the tauon) albeit with a 3491 mass factor instead. This gives rise to the concept of Lepton universality. The reason why the number of lepton generations is three remains unclear. Muons are of particular interest because:


\begin{itemize}
\item Parity violation in the weak sector is a particularly important part of modern physics: nature isn't invariant to spatial transformations, this is one of the results of the experiment.

\item A fairly accurate measurement for the weak coupling constant can be performed along with the muon's lifetime. The measurement of the anomalous magnetic moment of the muon, the so-called `g-2 factor' is an important test for the validity of QED.

\item Muons are the main component of sea-level cosmic radiation. The radiation density remains low however: the verticle flux of muons above \SI{1}{GeV/c}
\end{itemize}

